\textcolor{red}{\textbf{Abstract:} These are some basic notations and concepts which are used throughout this compilation.}

\section{Computational Complexity: P, NP, Proofs and Optimization}
We encounter various types of computational problems. The easier set of problems are \textbf{decision problems}, i.e. problems with an answer of the form $/{YES, NO/}$. Most other types of computational problems can be reduced to similar questions involving decision problems. A decision problem can be identified by the subset of inputs that have answer $YES$ and we can thus denote it by $x \in P$ instead of $P(x) = YES$ and similarly $x \notin P$ in place of $P(x) = NO$. This reduces decision problems to a set of \textbf{membership queries} in a given set. We refer to an answer $YES$ to an input as \textbf{accepting} the input and an answer $NO$ as \textbf{rejecting} the same input.

 \subsection{PolyTime Complexity Class}
 The poly-time complexity class consists of all the problems that have  a solution time that can be decided as  a polynomial function of of the size of the problem by a deterministic algorithm. We assume that \textbf{efficient algorithms} are those which use at most \textbf{polynomial} amount of computational resources with the main resource we care about is the worst-case \textbf{running time} of the algorithm with respect to it's input size. Class P (Polynomial time) problems can be solved in $O(N^C)$ time where C is any constant and N in s the input size of the problem. A problem belonging to class P can be solved in number of steps bounded by some power of the problem's size. The above assumption about the efficiency of polynomial time algorithms is known as \textbf{Cobham's thesis}. We can formally define polynomial time complexity class as follows:
 
 There exists an efficient algorithm $\textbf{A}$ such that for all inputs $x$:
 \begin{enumerate}
     \item if $P(x) = YES$ then $A(x) = YES$,
     \item if $P(x) = NO$ then $A(x) = NO$.
 \end{enumerate}
 
 \subsection{Non-PolyTime Complexity Class}
 The non-polytime complexity class consists of all the problems that require a non-deterministic algorithm to be solvable in polynomial time. Given a solution to the problem, however, we can determine whether it is correct or not in polynomial time. \textbf{NP is the class of problems which have efficient verifiers} i.e. there is a polynomial time algorithm that can verify if a given solution is correct. We can formally define non-polynomial time complexity class as follows:
 
There exists an efficient algorithm $\textbf{V}$ or the verifier such that for all inputs $x$:
\begin{enumerate}
    \item if $Q(x) = YES$ then there is a proof $y$ such that $V(x,y) = YES$
    \item if $Q(x) = NO$ then for all proofs $y$, $V(x,y) = NO$
\end{enumerate}

A sound verifier is one which can not be tricked into accepting a false positive proof as correct. It does not accept any proof wen the answer is $NO$. On the same note, a verifier is complete if it can accept atleast one proof when the answer is $YES$. The verifier $V$ gets two inputs:

\begin{enumerate}
    \item $x$: the original input for $Q$, and
    \item $y$: a suggested proof for $Q(x) = YES$
\end{enumerate}

We require proof $y$ to be readable in polynomial time by the verifier $V$. If y is short saying that V is efficient in x is the same as saying that V is efficient in x and y (because the size of y is bounded by a fixed polynomial in the size of x).

\subsection{P \subset NP}
As we have \textbf{P = efficiently solvable} and \textbf{NP = efficiently verifiable}, we can say that the set of $P = NP$ if efficienty solvable problems are efficiently verifiable. However, we can easily state that $P \subset NP$ as any problem that is solvable in polynomial time can be eaily verifiable in polynomial time. Suppose problem $p_1$ is in $P$ and you want to show that $p_1$ is also in $NP$.

Given a particular problem instance $i$ of $p_i$ and a candidate solution $c_i$, the following algorithm will decide if $c_i$ is really the correct solution:
\begin{enumerate}
    \item Since $p_1$ is in $P$, we have a polynomial time algorithm that can be used to determine the answer for $p_i$. Run this algorithm on $p_i$ and get a yes or no answer.
    \item Returns this as the answer.
\end{enumerate}