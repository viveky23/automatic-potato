\textcolor{red}{\textbf{Abstract:} }
\section{Probabilistically Checkable Proofs and Robustness}
We randomly need to select 3-bits according to some specific distribution and read only them and based on their value decide if the proof is good or bad. You can have some error but if you read the proof enough times the probability of error (i.e. of accepting a bad proof, is at most 1/2) goes down to $2^{-k}$ after reading $3k$ bits of the proof. 

\textbf{What is mathematical proof?} Any thing that can be verified by a rigorous procedure ie. an algorithm. A theorem can be seen as any decision problem and a solution to that problem can be seen as its proof. The difference in the solution and an equation exists form computational complexity POV as once you have an equation it might be hard to search for a solution but if you have a solution it is relatively easier to verify its correctness.

Given a system of linear equations we can determine the solution to the system using Gaussian Elimination in polynomial time. However, given an overdetermined set of equations of which not all are simultaneously satisfiable, then for any constant $\epsilon > 0$, given a $(1 − \epsilon)$-satisfiable linear system over the rationals, it is NP-hard to find an assignment to the variables that satisfies even a fraction $\epsilon$ of the equations. Given a 3-coloring problem with an input graph and the goal to decide if there is a way to 3-color the graph. However, it is very easy to check a proof.


 