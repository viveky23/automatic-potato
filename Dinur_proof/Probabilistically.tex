\textcolor{red}{\textbf{Abstract:} }

In order to verify some property using PCP, the prover needs to present a proof $P$ which is \textbf{easily testable}. Testing can be done by querying the proof at $q(n)$ points where $q(n)$ is a small number. PCP needs to satisfy the following properties: 
\begin{enumerate}[nolistsep]
    \item If the property actually holds, the verifier should always (or almost always) accept.
    \item If the property doesn't hold, then the verifier would notice with some constant probability.
\end{enumerate}

To account for lengthy proofs, as the verifier will only be looking at a small number- $q(n)$ points, randomness in choosing points is encouraged so as to make it harder to fool the verifier. The verifier chooses $r(n)$ random coins, and based on them queries the proof at $q(n)$ points. Then she applies a test which always succeeds in YES instances (the property holds), but fails with constant probability (over the coin tosses) in NO instances (the property doesn't hold). Since there are $r(n)$ coins, there are $2^{r(n)}$ different outcomes for the coins, and so the verifier in total queries the proof in at most $2^{r(n)}q(n)$ different places.

We use PCP because we can encode the verifier as a CNF whose variables are bits of the proof, for all $2^{r(n)}$ coin tosses, there exist clauses encoding that the $q(n)$ bits that were sampled, do pass the verifier's test. In case of a positive, $YES$ instance the CNF is always satisfiable and in case of a negative, $NO$ instance we see failure with some constant probability leading to a fraction of the clauses always being unsatisfied. 

\section{Basics}
We describe a verifier such that for any yes instance of $X$, there exists a certificate that the verifier will accept, and for any no instance of $X$, no such certificate exists. The running time of the verifier (and hence the size of the certificate) must be polynomial. This is formally represented as a function $V: \{0,1\}* \times \{0,1\}* \rightarrow \{0,1\}$ where, $V(T,\pi) = 1$ implies $\pi$ is the proof of assertion $T$, making it a proof. Classical verifiers/algorithms are enhanced by giving them access to random strings and oracles. Oracles will be functions $O: Q \rightarrow A$ with $Q$ being a countable set and $A$ being a finite set. In the query model, you are given
black-box access to some function $f$. You have to answer some question about $f$. 

Instead of measuring the time complexity of your algorithm, we measure the query complexity: the number of queries it makes to f. A probabilistic algorithm $A(·, ·)$ is an algorithm that takes two inputs $x$ and $r$, where $x$ is an instance of some problem that we want to solve, and $r$ is the output of a random source. A random source is an idealized device that outputs a sequence of bits that are uniformly and independently distributed. The output of a probabilistic oracle algorithm $A$ on input $x$, random string $R \in \{0,1\}*$ and access to oracle $O$ will be denoted as $A^O(x,R)$.

\textbf{Definition 1} \textit{For functions r,q,a : $\mathbb{Z}^+ \rightarrow \mathbb{Z}^+$ an (r,q,a) restricted PCP verifier is a probabbilistic oracle algorithm V that on input $x \in \{0,1\}^n$ expects a random string $R \in \{0,1\}^{r(n)}$ and queries an oracle $\pi : \mathbb{Z}^+ \rightarrow \{0,1\}^{a(n)}$ at most q(n) times and computes a "Boolean verdict" $V^{\pi}(x,R) \in \{0,1\}$.}

\textbf{Definition 2} \textit{For positive constant $0 \leq s \leq 1$, we can say that an (r,q,a)-restricted PCP verifier v accepts a language $L \subseteq \{0,1\}^*$ with soundness s if for every $x \in \{0,1\}^n the following hold:$}

\textbf{Completeness} \textit{If $x \in L$ then there exists a $\pi : \mathbb{Z}^+ \rightarrow \{0,1\}^{a(n)}$ such that for every $R$, $V^{\pi}(x:R) = 1$}

\textbf{Soundness} \textit{If $x \notin L$ then for every $\pi : \mathbb{Z}^+ \rightarrow \{0,1\}^{a(n)}$} it is the case that $Pr_R[\pi : \mathbb{Z}^+ \rightarrow \{0,1\}^{a(n)}] \leq s$

\textit{By $PCP_s[r,q,a]$ we denote a class of all languages L such that there exists an (r,q,a) restricted PCP verifier accepting L with soundness s.} The above definition is called a non-adaptive verifier. An adaptive verifier accesses y one bit at a time and may adapt its computation according to the read bit. In our definition V has a fixed predicate , and simply outputs that predicate in the final step. The length of the proof $\pi$ is at most $2^rq$ since there are $2r$ possible random strings of length $r$ and, $q$ possible accesses for each string. Therefore if $r$ is $c log n(n = |x|)$ for some constant $c$, the proof $y$ is of length at most $n^cq$, i.e. polynomial in n.

The PCP theorem gives a PCP for 3SAT with $q(n) = O(\log n)$ and $r(n) = O(1)$. That means that given a 3CNF, we can come construct a new CNF of length $O(2^{q(n)}) = n^{O(1)}$ using the construction outlined in the preceding paragraph such that (i) if the original 3CNF formula is satisfiable, so is the new one, (ii) otherwise, at most $1-\epsilon$ fraction of the clauses can ever be satisfied in the new CNF. That gives a hardness of approximation result for 3SAT. 

Using Raz's parallel repetition theorem, Håstad obtained a construct in which in case (ii), at most $7/8+\epsilon$ of the clauses can be satisfied, thus obtaining an optimal inapproximability result. (It's optimal since a random assignment satisfies $7/8$ of the clauses; this algorithm can be derandomized using the method of conditional expectations.)
