\begin{titlepage}
\includegraphics[width=30mm]{Images/NYU-Logo.png}\\
\vspace{2cm}
Master's Programme in Computer Science
    \begin{center}
        \vspace*{1cm}
        
        {\LARGE \textbf{Probabilistically Checkable Proofs}}
        
        \vspace{0.5cm}
        Notes and Paper Reviews
        
        \vspace{1.5cm}
        by \\
        \vspace{1.5cm}
       $[$Vivek Yadav \url{v.yadav@nyu.edu}$]$
                
        \vspace{1.0cm}
    \end{center}

\noindent{
\textcolor{red}{{\bf Abstract} (A probabilistically checkable proof system (PCP) for a language consists of a probabilistic polynomial-time verifier having direct access to individual bits of a binary string. This string (called oracle) represents a proof, and typically will be accessed only partially by the verifier. Queries to the oracle are positions on the bit string and will be determined by the verifier’s input and coin tosses (potentially, they might be determined by answers to previous queries as well). The verifier is supposed to decide whether a given input belongs to the language. If the input belongs to the language, the requirement is that the verifier will always accept given access to an adequate oracle. On the other hand, if the input does not belong to the language, then the verifier will reject with probability at least 1/2, no matter which oracle is used.)
}
}

\end{titlepage}